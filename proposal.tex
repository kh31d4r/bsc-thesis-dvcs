\documentclass{article}

\title{Topic Proposal Summary: \\ Distributed Version Control Systems}
\author{
Olof Johansson \\\small olof@ethup.se \\
\and Daniel Persson \\\small daniel@silvertejp.org
}

\begin{document}

\maketitle

\section{Background}

Our proposed topic for the thesis is distributed version control systems
(DVCS) and feasible workflows relating to this. Version control and
source control management (SCM) is an important task in software
engineering, and the tools used determines if it will become an
obstruction rather than the convenience that it probably was intended to
be. One of the current trends within source control management is to
have the version control system be distributed\cite{sink10}, without a
central authority (at least not a central authority imposed by the
system itself).

According to a survey among users of the Eclipse IDE\cite{eclipse10}, the 
rise of one such system --- git --- has increased from 2.4\% to 6.8\% from 
2009 to 2010. And it is not the only DVCS with a steep increase in usage: 
Mercurial increased from 1.1\% to 3\% over the same period of time. 
DVCS tools are especially popular within the open source community, 
with large projects, such as the Linux kernel\cite{kernel-git}, 
Gnome\cite{gnome-git} and Firefox\cite{firefox-hg} using them.

It is not unusual to use distributed version control system in a 
centralized way, but the opportunity to use it decentralized is 
interesting. Within our project we are trying to enforce a distributed 
workflow, and it would be interesting to learn what the benefits are, 
what are the obstacles, and how we are going to implement this. It is
important to evaluate the benefit from this workflow and document the
experience gained from its use in the project.

\section{Research Questions}

We will try to answer the following research questions:

\begin{itemize}
	\item How do the developers adapt to this workflow? 
	\item How does this workflow affect software quality?
\end{itemize}

The outcome will be an evaluation of distributed VCS workflows, with
interviews of developers and data from the project. The data will
consist of repository metadata (commits, commitdiffs, etc.). This
contribution will be useful as a prestudy for others interested in
deploying this type of workflow and as a reference for organizations or
researchers interested in deploying other type of workflows.

\section{Research Methodology}

\subsection{Post-mortem analysis}
To answer our research questions, we will conduct post-mortem
interviews with project team members and participants in other
projects within the same course. We will question them on their
experience with version control and how they perceive version control
concepts. In our project we will interview every team member in their
different roles. As stated before we will also interview members from
other projects, specifically people with prior VCS experience to get a
reference point from a non-distributed perspective.

We want to find the strengths and weaknesses of the workflow described
above through the interviews, and then back these findings up with
data analysis of meta data associated with the VCS repository
(e.g. numberof commits, number of commits per author, number of
merges, size of commits) and literature studies within the field of
version control and configuration management.

\subsection{Literature search strategies}

Proposals for search items:
\begin{itemize}
  \item VCS, Version Control
  \item DVCS, Distributed Version Control
  \item SCM, Source Control Management
  \item Git, Subversion, Mercurial, Perforce, Darcs, CVS, BitKeeper
  \item Checkin, Checkout, Commit, Diff, Branch, Merge, Rebase
  \item Configuration Management
\end{itemize}

We will combine items from this list into relevant search strings to
look for in Google Scholar, IEEE Xplore, Elin and other relevant web
archives.  To filter relevant articles from these results we will
primarily look at the articles' abstract and conclusions. From the
articles we find relevant we will also look at the references and
investigate if they are relevant for the topic.

\bibliographystyle{plain}
\bibliography{references}

\end{document}
